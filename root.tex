\documentclass[11pt]{book}
\usepackage{proof,latexsym,xspace,amsmath,amssymb,graphicx,color,enumerate,imakeidx}
\usepackage{kotex,setspace,varwidth}

\renewcommand{\contentsname}{차례}
\renewcommand{\chaptername}{}
\renewcommand{\figurename}{그림}
\renewcommand{\appendixname}{}
\renewcommand{\bibname}{}
\renewcommand{\indexname}{쉽게 쓰는 전문용어}

\input{macroA}\input{macroB}\input{macroC}
\definecolor{dark-gray}{gray}{0.5} % 0-1, the higher the lighter
\definecolor{light-gray}{gray}{0.8} % 0-1, the higher the lighter

% index both an english and its korean terms
\newcommand{\tindex}[2]{\index{#1!#2}\index{#2!#1}}
% jargon \jar{english}{korean}
\newcommand{\jar}[2]{{#2}$_{\textit{\footnotesize #1}}$\tindex{#1}{#2}}
% jargon inside footnote
\newcommand{\jarfoot}[2]{{#2}$_{\textit{\scriptsize #1}}$\tindex{#1}{#2}}
% jargon for glossary chapter, to escape the indexing
\newcommand{\jarg}[2]{{#2}$_{\textit{\footnotesize #1}}$}

%\makeindex[columnsep=20pt, intoc, noautomatic]
\makeindex[columns=1]

\title{}
\author{수집\(\cdot\)제안\(\cdot\)편집\ \ \  이 광근
}
\date{}

\begin{document}
\maketitle

\tindex{abbreviation}{줄임말}
\tindex{abduction}{귀추}
\tindex{abduction}{앱덕}
\tindex{abduction}{원인 짐작하기}
\tindex{abstract interpretation}{요약해석}
\tindex{abstract semantics}{요약된 의미구조}
\tindex{abstract syntax}{핵심 문법구조}
\tindex{abstract type}{구현된 속사정이 감추어진 타입}
\tindex{abstract type}{속내용이 감추어진 타입}
\tindex{abstract type}{추상적인 타입}
\tindex{abstraction}{요약}
\tindex{abstraction}{속내용 감추기}
\tindex{abstraction}{핵심 드러내기}
\tindex{allocation}{메모리 할당}
\tindex{abstraction hierarchy}{속내용 감추며 차곡차곡 쌓기}
\tindex{application}{호출}
\tindex{applicative language}{값 중심의 언어}
\tindex{array row}{배열 내용}
\tindex{assignment}{메모리에 쓰기}
\tindex{association list}{관계 리스트}
\tindex{associativity}{방향성}
\tindex{associativity}{결합법칙}
\tindex{attribute grammar}{할일이 딸려있는 문법}
\tindex{attribute grammar}{속성 문법}
\tindex{axiomatic thoery}{엄밀한 논리 시스템}

\tindex{polymorphic}{모양이 다양한}
\tindex{polymorphic}{다형의}
\tindex{binary}{두개의}
\tindex{bind}{묶다}
\tindex{bind}{이름짓다}
\tindex{bind}{정의하다}
\tindex{binding}{명명하기}
\tindex{binding}{이름짓기}
\tindex{binding}{정의하기}
\tindex{Boolean expression}{부울식}
\tindex{bounded probabilistic polynomial}{오류율을 잡아둘 수 있는 확률형 다항}
\tindex{bottom}{바닥}
\tindex{built-in}{붙박이}
\tindex{built-in}{이미 있는}

\tindex{calculus}{계산법}
\tindex{calculus}{셈법}
\tindex{call by name}{식전달 호출}
\tindex{call by reference}{주소전달 호출}
\tindex{call by value}{값전달 호출}
\tindex{Cartesian product}{데카르트 곱}
\tindex{Cartesian product}{완전곱}
\tindex{case expression}{선택식}
\tindex{closure conversion}{함수 변환}
\tindex{closure conversion}{함수의 자유변수를 없애주는 변환}
\tindex{closure conversion}{함수가 인자를 통해서만 외부와 소통하게 하는 변환}
\tindex{compilation unit}{컴파일 단위}
\tindex{compilation unit}{번역 단위}
\tindex{compilation}{언어의 기계어 변환}
\tindex{compilation}{프로그램 번역 }
\tindex{complete partially ordered set}{완전히 부분 순서를 가지는 집합}
\tindex{complete}{완전한}
\tindex{completeness}{완전성}
\tindex{computation strategy}{계산 방식}
\tindex{computation strategy}{계산 전략}
\tindex{computation}{계산}
\tindex{computational complexity}{계산 복잡도}
\tindex{computational learning theory}{계산 학습 이론}
\tindex{conjunction}{그리고-식}
\tindex{conjunctive normal form}{그리고-조합 바른 식}
\tindex{conjunctive normal form}{그리고-조합 표준형}
\tindex{convex programming}{볼록 프로그래밍}
\tindex{concrete syntax}{구체적 문법 구조}
\tindex{consistency}{일관성}
\tindex{constant}{상수}
\tindex{constraint expression}{제약식}
\tindex{constraint}{제약}
\tindex{construction}{구성 방법}
\tindex{constructive type}{생성적인 타입}
\tindex{constructor bind}{데이타 구성자 정의}
\tindex{constructor description}{데이타 구성자 접속방안}
\tindex{constructor}{데이타 구성자}
\tindex{context}{문맥}
\tindex{context}{환경}
\tindex{continuation passing style transform}{계산과정 전달 변환}
\tindex{continuation passing style}{계산과정을 전달하는}
\tindex{continuation passing style}{앞으로 할 일을 함수로 정리해서 전달해주는}
\tindex{continuation}{앞으로 할 계산}
\tindex{continuation}{앞으로 할 일}
\tindex{control flow analysis}{함수 흐름 분석}
\tindex{control flow analysis}{실행 흐름 분석}
\tindex{control structure}{실행 순서}
\tindex{convergence}{수렴성}
\tindex{convergent}{수렴하는}
\tindex{correctness}{올바름}
\tindex{curried application}{커리형 함수의 적용}
\tindex{curried function}{커리형 함수}
\tindex{currified}{커리화한}
\tindex{currying}{커링}

\tindex{dangling pointer}{오염된 메모리}
\tindex{dangling pointer}{잘못된 포인터}
\tindex{dangling pointer}{재생된 메모리}
\tindex{dangling pointer}{재활용된 메모리}
\tindex{data constructor}{데이타 구성자}
\tindex{data constructor}{자료 구성자}
\tindex{data description}{데이타 타입 접속방안}
\tindex{data structure}{데이타 구조}
\tindex{data structure}{자료 구조}
\tindex{de-sugar}{설탕 구조를 풀다}
\tindex{de-sugar}{설탕구조를 녹이다}
\tindex{dead code}{계산되지 않는 코드}
\tindex{dead code}{사용되지 않는 값}
\tindex{dead code}{사용되지 않는 코드}
\tindex{decision problem}{결정문제}
\tindex{declaration}{선언}
\tindex{deduction}{디덕}
\tindex{deduction}{반드시 이끌기}
\tindex{deduction}{연역}
\tindex{deep neural net}{깊은 신경망}
\tindex{deep neural net}{딥뉴럴넷}
\tindex{delayed evaluation}{최대한 미루어 계산하는 방법}
\tindex{denotational semanitcs}{함수형 의미구조}
\tindex{denotational semanitcs}{고정점 방식의 의미구조}
\tindex{denotational semanitcs}{수학적인 의미구조}
\tindex{destructive}{파괴적인}
\tindex{determinisitc}{모든게 정해진}
\tindex{determinisitc}{확실한 연산만 있는}
\tindex{digit}{숫자}
\tindex{disjunction}{또는-식}
\tindex{disjunctive normal form}{또는-조합 바른 식}
\tindex{disjunctive normal form}{또는-조합 표준형}
\tindex{dynamic scoping}{이름의 유효범위가 실행 중에 결정되는}
\tindex{dynamic scoping}{실행중에 드러나는 이름의 실체}
\tindex{dynamic semantics}{프로그램의 실행}

\tindex{environment enrichment}{기획 환경의 적응}
\tindex{environment function}{환경 함수}
\tindex{environment unroll}{실행환경 펼치기}
\tindex{environment}{환경}
\tindex{equational reasoning}{같은것들을 궁리해가는 }
\tindex{equational reasoning}{등치만들기 논법}
\tindex{error}{오류}
\tindex{evaluation by value}{적극적인 계산법}
\tindex{evaluation strategy}{계산 방식 }
\tindex{evaluation strategy}{계산 전략}
\tindex{evaluation strategy}{계산법}
\tindex{evaluation strategy}{시험 전략}
\tindex{evaluation}{계산}
\tindex{evaluation}{수행}
\tindex{exception bind}{예외상황 정의}
\tindex{exception description}{예외상황 접속방안}
\tindex{exception}{예외상황}
\tindex{explicit}{드러난}
\tindex{expression}{프로그램식}

\tindex{factorial}{계승}
\tindex{field}{필드}
\tindex{finiteness}{유한성}
\tindex{first-order equational logic}{단순 등식 논리}
\tindex{free identifier}{묶이지 않은 이름}
\tindex{free identifier}{자유로운 이름}
\tindex{free type name}{묶이지 않은 타입 이름}
\tindex{free variable}{묶이지 않은 변수}
\tindex{free variable}{자유로운 변수}
\tindex{free variable}{자유로운 이름}
\tindex{free variable}{자유변수}
\tindex{function abstraction}{함수}
\tindex{function abstraction}{함수 추상화}
\tindex{function abstraction}{함수로 만들기}
\tindex{function application}{계산}
\tindex{function application}{함수 적용}
\tindex{function application}{함수 적용식}
\tindex{function application}{함수 호출}
\tindex{function argument}{함수의 인자}
\tindex{function expression}{함수 표현식}
\tindex{function expression}{함수식}
\tindex{function}{함수}
\tindex{functional language}{값 중심의 언어}
\tindex{functional language}{함수형 언어}
\tindex{functional language}{함수 중심의 언어}
\tindex{functional programming}{값 중심의 프로그래밍}
\tindex{functional style}{값 중심 스타일}
\tindex{functional style}{함수 중심 스타일}
\tindex{functional}{함수}
\tindex{functor signature instantiation}{모듈함수 타입의 실현}
\tindex{functor}{모듈함수}
\tindex{fuzzing}{마구잡이 깨기}
\tindex{sw fuzzing}{마구잡이 sw깨기}    

\tindex{garbage collection}{메모리 재활용}
\tindex{generative}{생성적}
\tindex{grammar}{문법}
\tindex{grammar}{문법 기술}

\tindex{halting problem}{멈춤문제}
\tindex{heap profiler}{메모리 계측기}
\tindex{high-order function}{고차 함수}
\tindex{high-order function}{함수를 주고 받는 함수}
\tindex{higher-order and typed}{고차 타입을 갖춘}

\tindex{identifier}{식별자}
\tindex{identity function}{항등 함수}
\tindex{imperative language}{기계중심의 언어}
\tindex{imperative language}{메모리 중심의 언어}
\tindex{imperative language}{명령형 언어}
\tindex{imperative language}{행동지침형 언어}
\tindex{incomplete}{불완전한}
\tindex{incomplete}{빠뜨리는게 있는}
\tindex{incompleteness theorem}{불완전성 정리}
\tindex{induction}{인덕}
\tindex{induction}{짐작해서 이끌기}
\tindex{induction}{귀납}
\tindex{infix}{새치기}
\tindex{insertion sort}{삽입 정렬}
\tindex{interface}{접속 방안}
\tindex{interface}{접속 형태}
\tindex{interface}{사용법}
\tindex{interpretation}{실행}
\tindex{isomorphism}{같은 형태}
\tindex{iterative}{반복적}

\tindex{join operator}{결합 연산자}

\tindex{label row}{레코드 내용}
\tindex{lattice}{격자}
\tindex{lattice}{래티스}
\tindex{lazy evaluation}{값 계산을 최대한 미루는}
\tindex{lazy evaluation}{소극적 계산법}
\tindex{lazy evaluation}{지연 계산 }
\tindex{lazy evaluation}{필요한 때만 값을 계산하는}
\tindex{leaf}{말단노드}
\tindex{lexical conventions}{어휘 만드는 방법}
\tindex{lexical scope}{사전적인 유효 범위}
\tindex{lexicographic order}{사전적 순서}
\tindex{lexicographic order}{사전적 순서 관계}
\tindex{linear function}{직선 함수}
\tindex{list}{리스트}
\tindex{local definition}{우물안 정의}
\tindex{local definition}{지역적인 정의}
\tindex{logical relation}{논리적 관계}

\tindex{machine learning}{기계 학습}
\tindex{match}{패턴 맞추기}
\tindex{memory function}{메모리 함수}
\tindex{memory leak}{메모리 출혈}
\tindex{metalanguage}{언어를 기술하는 언어}
\tindex{module}{모듈}
\tindex{mono-variant analysis}{다수의 프로그램 흐름을 하나로 요약하는 분석}
\tindex{mono-variant analysis}{다대일 분석} 
\tindex{mono-variant analysis}{단일성 분석}
\tindex{mutual recursive}{서로 맞물려서 호출하는}
\tindex{mutual recursive}{서로 호출하는}

\tindex{negation}{뒤집기}
\tindex{network}{네트웍}
\tindex{node}{노드}
\tindex{non-deterministic}{운에 기대면}
\tindex{non-deterministic polynomial}{운에 기대면 다항시간 안에 풀리는}
\tindex{non-expansive}{메모리 반응을 일으키지 않는}
\tindex{normal form}{바른 모습}
\tindex{normal form}{바른 꼴}
\tindex{normal form}{표준형}

\tindex{object}{물건}
\tindex{object-oriented language}{물건 중심의 언어}
\tindex{operational semantics}{실행과정을 드러내는 의미구조} 
\tindex{operator}{연산자}
\tindex{or-pattern}{무더기 패턴}
\tindex{ordered relation}{순서 관계}
\tindex{overflow}{넘침}

\tindex{Probably Approximately Correct, PAC}{얼추거의맞기}
\tindex{parameter}{인자}
\tindex{parameterized module}{일반화된 모듈}
\tindex{parity function}{홀짝 함수}
\tindex{partial function}{일부만 정의된 함수}
\tindex{pattern match}{패턴에 맞추기}
\tindex{pattern match}{패턴에 대보기}
\tindex{pattern row}{레코드 패턴}
\tindex{pattern}{패턴}
\tindex{poly-variant analysis}{다형성을 가지는 분석}
\tindex{poly-variant analysis}{다대다 분석}
\tindex{poly-variant analysis}{다수의 프로그램 흐름을 하나이상으로 요약하는 분석} 
\tindex{polymorphic function}{다형 함수}
\tindex{polymorphic function}{인자 타입에 상관없는 함수}
\tindex{polymorphic}{다변형}
\tindex{polymorphic}{다형}
\tindex{polymorphic}{여러 모양의}
\tindex{polymorphic}{여러 타입을 가지는}
\tindex{polymorphism}{다형성}
\tindex{postfix}{뒤에 붙는}
\tindex{precedence}{우선순위}
\tindex{predicate logic}{모든-어떤 논리}
\tindex{predicate logic}{술어 논리}
\tindex{prefix}{앞에 붙는}
\tindex{primitive recursive function}{단순한 재귀 함수 }
\tindex{primitive recursive function}{원시적인 재귀 함수}
\tindex{primitive}{기본}
\tindex{principal type}{가장 일반적인 타입}
\tindex{principal type}{대표 타입}
\tindex{programming language}{프로그래밍 언어}

\tindex{ramdomization}{무작위}
\tindex{ramdomized algorithm}{무작위 알고리즘}
\tindex{record}{레코드}
\tindex{recursive function}{자기 호출 함수}
\tindex{recursive function}{재귀함수}
\tindex{recursive function}{자기자신을 부르는 함수}
\tindex{recursive primitive definition}{원시적 자기참조 정의}
\tindex{recursive}{자기자신을 부르는}
\tindex{recursive}{자기호출}
\tindex{reduction}{계산}
\tindex{reduction}{수행}
\tindex{reduction}{줄이기}
\tindex{reference manual}{참고서}
\tindex{reference}{메모리 주소}
\tindex{reasoning}{이치따지기}
\tindex{rewrite rule}{다시 쓰기 규칙}
\tindex{rewrite semantics}{다시 쓰기를 이용한 의미 구조}
\tindex{rewrite}{다시 쓰기}

\tindex{scheme}{틀}
\tindex{scope}{유효범위}
\tindex{skolemization}{안전하게 정량자 제거하기}
\tindex{semantics}{의미}
\tindex{semantics}{의미구조}
\tindex{sequence}{나열식}
\tindex{side-effect}{메모리 반응 }
\tindex{side-effect}{수반되는 반응}
\tindex{side-effect}{함께오는 반응}
\tindex{signature bind}{모듈타입 정의}
\tindex{signature instantiation}{모듈 타입의 실현}
\tindex{signature matching}{모듈 접속}
\tindex{signature}{모듈타입}
\tindex{soundness}{안전성 }
\tindex{specification}{접속 방안}
\tindex{static analysis}{프로그램 분석}
\tindex{static scope}{정적인 유효 범위}
\tindex{static scoping}{이름의 유효범위가 미리 결정되는}
\tindex{static scoping}{실행전에 결정되는 이름의 실체 }
\tindex{static semantics}{프로그램의 기획}
\tindex{static type synthesis}{타입 유추}
\tindex{strict evaluation}{일단 값을 계산하고 보는}
\tindex{strict evaluation}{적극적 계산법}
\tindex{structural}{구조적}
\tindex{structure bind}{모듈 정의}
\tindex{structure description}{모듈 접속방안}
\tindex{structure expression}{모듈식}
\tindex{structure}{모듈}
\tindex{substitution}{바꿔치기}
\tindex{substitution}{치환 함수}
\tindex{symbol}{기호}
\tindex{syntactic constraint}{문법적인 제약}
\tindex{syntactic sugar}{설탕구조}
\tindex{syntax analysis}{문법 구조 분석}
\tindex{syntax}{문법 구조}

\tindex{tail recursive}{마지막에 자기자신을 부르는}
\tindex{tail recursive}{끝 재귀호출}
\tindex{tail recursive}{자기 호출이 마지막인}
\tindex{top declaration}{가장 위의 선언}
\tindex{top-level declaration}{가장 위의 선언}
\tindex{total function}{모두가 정의된 함수}
\tindex{tree}{가지구조}
\tindex{tree}{나무구조}
\tindex{truely recursive function}{진심인 재귀 함수}
\tindex{truely recursive function}{자기참조 없이는 정의할 수 없는 함수}
\tindex{tuple}{튜플}
\tindex{type abbreviation}{타입 줄임말}
\tindex{type bind}{타입 정의}
\tindex{type construct}{타입식}
\tindex{type constructor}{타입}
\tindex{type constructor}{타입 구성자}
\tindex{type description}{타입 접속방안}
\tindex{type expression}{타입식}
\tindex{type inference}{타입 유추}
\tindex{type realization}{타입 실현}
\tindex{type scheme generalization}{타입 틀의 적응}
\tindex{type scheme}{타입 틀}
\tindex{type structure enrichment}{타입 구조의 적응}
\tindex{type structure}{타입 구조}
\tindex{type variable}{타입 변수}
\tindex{type}{타입}
\tindex{typing rule}{타입만들기 규칙}

\tindex{unary}{인자가 하나인}
\tindex{uncurrying}{언커링}
\tindex{undecidable}{결정할 수 없는}
\tindex{undecidable}{컴퓨터로는 불가능한}
\tindex{unification}{동일화}
\tindex{unification}{타입을 같게 하는}
\tindex{universal machine}{보편만능 기계}

\tindex{value bind}{값 정의}
\tindex{value description}{값 접속방안}
\tindex{value type enrichment}{값 타입의 적응}
\tindex{value}{값}
\tindex{variable}{변수}

\tindex{weak consistency}{약한 일관성}
\tindex{well founded}{밑바닥이 튼실한 }
\tindex{well founded}{바닥이 갖추어진 }
\tindex{well founded}{올바르게 기초하였다}
\tindex{well-formed}{제대로 구성된}
\tindex{wild pattern}{임의 패턴}



\setstretch{1.0}
%\clearpage{\pagestyle{empty}\cleardoublepage}
\printindex

\end{document}
